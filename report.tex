\documentclass[10pt]{article}
%Gummi|063|=)
\usepackage{times}
\usepackage[T1]{fontenc}
\usepackage[english]{babel}
\usepackage{geometry}
 \geometry{
 a4paper,
 total={170mm,257mm},
 left=20mm,
 top=20mm,
 }
\usepackage{cite}

\title{\textbf{Performance Analysis Of MPI Over Cluster}}
\author{Achuth PV\\
		Anirudh Rao\\
		Avishkar P}
\date{}
\begin{document}

\maketitle

\section{Introduction}

\section{Details of Infiniband Cluster}
\begin{itemize}
\item Number of machines: 14 -  2 masters, 12 slaves
\item OS: CentOS release 6.5 (Final)
\item CPU: Intel(R) Xeon(R) CPU E5-2670 0 @ 2.60GHz
\begin{itemize}
\item 2 CPUs: NUMA with 8 cores/each, 1 thread/core
\item Cache: L1d - 32K, L1i - 32K, L2 : 256K, L3 : 20480K
\end{itemize}
\item 64GB RAM
\item OpenMPI Version : 1.8.4
\item MVAPICH2 Version: 2.1
\item InfiniBand card: QLogic Corp. IBA7322 QDR InfiniBand HCA
\item Infiniband switch : QLogic 12200-18 36-ports
\item Link speed: 40 Gb/sec (4X QDR)
\item Ethernet controller: Intel Corporation I350 Gigabit Network Connection
\end{itemize}
For more details about CPU and memory, please refer Infiniband\textunderscore CPU\textunderscore details.txt and Infiniband\textunderscore Memory\textunderscore details.txt.

\section{Setting up of CUDA Cluster}
We setup a CUDA cluster of 4 machines. All the machines are Dell Optiplex 3020 with the following specifications
\begin{itemize}
\item 4 Machines
\item Ubuntu 12.04 64 bit
\item Intel(R) Core(TM) i3-4130 CPU @ 3.40GHz
\begin{itemize}
\item 2 cores, 2 threads each
\item Cache - 128kB L1, 512kB L2,  3MB L3 
\end{itemize}
\item 4 GB RAM
\item GeForce GT 640
\begin{itemize}
\item CUDA cores 384 
\item 2048 MB RAM
\item Memory clock rate: 891 Mhz
\item Memory bus width: 128-bit
\item GPU clock rate: 902 MHz 

\end{itemize}
\item CUDA Toolkit 6.5
\item 500 GB harddisk 
\item Ethernet controller: Realtek Semiconductor Co., Ltd. RTL8111/8168/8411
\item OpenMPI: 1.8.8
\item MVAPICH2: 2.1
\end{itemize}

To set up CUDA cluster, the following steps were followed
\begin{itemize}
\item A file based partitions is created~\cite{filebased}
\item A NFS server is set on the same machine, sharing the partition~\cite{nfssetup}
\item An user account 'mpiuser' is created on the machine with the nfs partition containing its home
\item The NFS partition is mounted on other machines on the cluster
\item The same 'mpiuser' with same gid and uid are created on all the machines, with the same home folder as earlier.
\item After logging in as 'mpiuser', ssh key is generated and added to authorized\textunderscore keys file
\item All the desired machines are added into known\textunderscore hosts 
\item Both mvapich2~\cite{mvapich} and openmpi~\cite{openmpi} versions supporting CUDA-aware MPI are downloaded, built and installed on home folder 
\end{itemize}

\begin{thebibliography}{9}
\bibitem{filebased} File based partition, \texttt{https://www.debuntu.org/\\how-to-create-a-filesystem-within-another-partitions-file}
\bibitem{nfssetup} NFS Setup, \texttt{https://www.digitalocean.com/community/tutorials/\\how-to-set-up-an-nfs-mount-on-ubuntu-12-04}
\bibitem{mvapich} MVAPICH2-2.1, \texttt{http://mvapich.cse.ohio-state.edu/static/media/\\mvapich/mvapich2-2.1-userguide.html}
\bibitem{openmpi} OpenMPI, \texttt{https://www.open-mpi.org/faq/?category=buildcuda}
\end{thebibliography}
\end{document}
